\documentclass{article}
\usepackage{arxiv}
\usepackage[utf8]{inputenc} 
\usepackage[T1]{fontenc}    
\usepackage{hyperref}       
\usepackage{url}            
\usepackage{booktabs}       
\usepackage{amsfonts}       
\usepackage{nicefrac}       
\usepackage{microtype}      
\usepackage{lipsum}	
\usepackage{graphicx}

\title{Radial Basis Networks for Global Optimization}

\author{ \hspace{1mm}Burak Bayramli  \\
	\texttt{bb@alumni.stevens-tech.edu} \\
}

\begin{document}
\maketitle

\begin{abstract}

\end{abstract}


\keywords{Radial basis networks, global optimization, initial point calculation}

\section{Introduction}
\lipsum[2]
\lipsum[3]


\section{Headings: first level}
\label{sec:headings}

\lipsum[4] See Section \ref{sec:headings}.

Peaks function

\subsection{Headings: second level}
\lipsum[5]
\begin{equation}
f(x_1,x_2) = 3 (1 - x_1)^2 e^{-x_1^2-(-x_2^2 + 1)^2} - 
10 \bigg( \frac{x_1}{5} - x_1^3-x_2^5 \bigg) e^{-x_1^2 -x_2^2} - 
\frac{1}{3} e^{-(x_1 + 1)^2 - x_2^2}
\end{equation}

\subsubsection{Headings: third level}
\lipsum[6]

\paragraph{Paragraph}
\lipsum[7]

\section{Examples of citations, figures, tables, references}
\label{sec:others}
\lipsum[8] \cite{kour2014real,kour2014fast} and see \cite{hadash2018estimate}.

The documentation for \verb+natbib+ may be found at
\begin{center}
	\url{http://mirrors.ctan.org/macros/latex/contrib/natbib/natnotes.pdf}
\end{center}
Of note is the command \verb+\citet+, which produces citations
appropriate for use in inline text.  For example,
\begin{verbatim}
   \citet{hasselmo} investigated\dots
\end{verbatim}
produces
\begin{quote}
	Hasselmo, et al.\ (1995) investigated\dots
\end{quote}

\begin{center}
	\url{https://www.ctan.org/pkg/booktabs}
\end{center}


\subsection{Figures}
\lipsum[10]
See Figure \ref{fig:fig1}. Here is how you add footnotes. \footnote{Sample of the first footnote.}
\lipsum[11]

\begin{figure}
	\centering
	\fbox{\rule[-.5cm]{4cm}{4cm} \rule[-.5cm]{4cm}{0cm}}
	\caption{Sample figure caption.}
	\label{fig:fig1}
\end{figure}

\subsection{Tables}
\lipsum[12]
See awesome Table~\ref{tab:table}.

\begin{table}
	\caption{Sample table title}
	\centering
	\begin{tabular}{lll}
		\toprule
		\multicolumn{2}{c}{Part}                   \\
		\cmidrule(r){1-2}
		Name     & Description     & Size ($\mu$m) \\
		\midrule
		Dendrite & Input terminal  & $\sim$100     \\
		Axon     & Output terminal & $\sim$10      \\
		Soma     & Cell body       & up to $10^6$  \\
		\bottomrule
	\end{tabular}
	\label{tab:table}
\end{table}

\subsection{Lists}
\begin{itemize}
	\item Lorem ipsum dolor sit amet
	\item consectetur adipiscing elit.
	\item Aliquam dignissim blandit est, in dictum tortor gravida eget. In ac rutrum magna.
\end{itemize}


\bibliographystyle{unsrt}
%\bibliography{references}  %%% Remove comment to use the external .bib file (using bibtex).
%%% and comment out the ``thebibliography'' section.


%%% Comment out this section when you \bibliography{references} is enabled.
\begin{thebibliography}{1}

	\bibitem{kour2014real}
	George Kour and Raid Saabne.
	\newblock Real-time segmentation of on-line handwritten arabic script.
	\newblock In {\em Frontiers in Handwriting Recognition (ICFHR), 2014 14th
			International Conference on}, pages 417--422. IEEE, 2014.

	\bibitem{kour2014fast}
	George Kour and Raid Saabne.
	\newblock Fast classification of handwritten on-line arabic characters.
	\newblock In {\em Soft Computing and Pattern Recognition (SoCPaR), 2014 6th
			International Conference of}, pages 312--318. IEEE, 2014.

	\bibitem{hadash2018estimate}
	Guy Hadash, Einat Kermany, Boaz Carmeli, Ofer Lavi, George Kour, and Alon
	Jacovi.
	\newblock Estimate and replace: A novel approach to integrating deep neural
	networks with existing applications.
	\newblock {\em arXiv preprint arXiv:1804.09028}, 2018.

\end{thebibliography}


\end{document}
