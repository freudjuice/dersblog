\documentclass[12pt,fleqn]{article}\usepackage{../../common}
\begin{document}
BFGS 

Newton y�ntemine g�re �zyineli d�ng� �u �ekilde belirtilir, 

$$
x_{i+1} = x_i - H_i^{-1} \nabla f (x_i)  
$$

ki $f$ fonksiyonunun $H$ Hessian matrisidir. Newton-imsi (quasi-Newton)
metotlarda ya $H_i$ matrisi bir di�er $A_i$ ile yakla��k olarak temsil
edilir (ya da $H_i ^{-1}$ matrisi $B_i$ ile). O zaman ustteki denklem yerine

$$
x_{i+1} = x_i - \lambda_i B_i \nabla f(x_i)
$$

ki $\lambda_i$ $s_i = -\lambda_i B_i \nabla f(x_i)$ y�n�nde bir optimal
ad�m uzunlu�udur.


[devam edecek]

\begin{minted}[fontsize=\footnotesize]{python}
import pandas as pd
import numpy as np
import matplotlib.pyplot as plt
import numpy.linalg as lin

eps = np.sqrt(np.finfo(float).eps)

def rosen(x):
    return 100*(x[1]-x[0]**2)**2+(1-x[0])**2

def rosen_real(x):
    gy =[-400*(x[1]-x[0]**2)*x[0]-2*(1-x[0]), 200*(x[1]-x[0]**2)]
    return rosen(x), gy

def linesearch_secant(f, d, x):
    epsilon=10**(-5)
    max = 500
    alpha_curr=0
    alpha=10**-5
    y,grad=f(x)
    dphi_zero=np.dot(np.array(grad).T,d)

    dphi_curr=dphi_zero
    i=0;
    while np.abs(dphi_curr)>epsilon*np.abs(dphi_zero):
        alpha_old=alpha_curr
        alpha_curr=alpha
        dphi_old=dphi_curr
        y,grad=f(x+alpha_curr*d)
        dphi_curr=np.dot(np.array(grad).T,d)
        alpha=(dphi_curr*alpha_old-dphi_old*alpha_curr)/(dphi_curr-dphi_old);
        i += 1
        if (i >= max) and (np.abs(dphi_curr)>epsilon*np.abs(dphi_zero)):
            print('Line search terminating with number of iterations:')
            print(i)
            print(alpha)
            break
        
    return alpha

def bfgs(x, func):
    
    H = np.eye(2)
    tol = 1e-20
    y,grad = func(x)
    dist=2*tol
    epsilon = tol
    iter=0;

    while lin.norm(grad)>1e-6:
        value,grad=func(x)
        p=np.dot(-H,grad)
        lam = linesearch_secant(func,p,x)
        iter += 1
        xt = x
        x = x + lam*p
        s = lam*p
        dist=lin.norm(s)
        newvalue,newgrad=func(x)
        y = np.array(newgrad)-grad
        rho=1/np.dot(y.T,s)
        s = s.reshape(2,1)
        y = y.reshape(2,1)
        tmp1 = np.eye(2)-rho*np.dot(s,y.T)
        tmp2 = np.eye(2)-rho*np.dot(y,s.T)
        tmp3 = rho*np.dot(s,s.T)
        H= np.dot(np.dot(tmp1,H),tmp2) + tmp3
        #print ('lambda:',lam)

    print (xt)
    print ('iter',iter)
\end{minted}


\begin{minted}[fontsize=\footnotesize]{python}
x=np.array([-1.0,0])
bfgs(x,rosen_real)    
\end{minted}

\begin{verbatim}
[1. 1.]
iter 19
\end{verbatim}

\begin{minted}[fontsize=\footnotesize]{python}
def _approx_fprime_helper(xk, f, epsilon):
    f0 = f(xk)
    grad = np.zeros((len(xk),), float)
    ei = np.zeros((len(xk),), float)
    for k in range(len(xk)):
        ei[k] = 1.0
        d = epsilon * ei
        df = (f(xk + d) - f0) / d[k]
        if not np.isscalar(df):
            try:
                df = df.item()
            except (ValueError, AttributeError):
                raise ValueError("The user-provided "
                                 "objective function must "
                                 "return a scalar value.")
        grad[k] = df
        ei[k] = 0.0
    return grad


def rosen_approx(x):
    g = _approx_fprime_helper(x, rosen, eps)
    return rosen(x),g

bfgs(x,rosen_approx)
\end{minted}

\begin{verbatim}
[0.99999552 0.99999104]
iter 19
\end{verbatim}


Kaynaklar 

Dutta, {\em Optimization in Chemical Engineering}

Zak, {\em An Introduction to Optimization, 4th Edition}

\end{document}


