\documentclass[12pt,fleqn]{article}\usepackage{../../common}
\begin{document}
Autograd ile Optimizasyon

Otomatik t�revin nas�l i�ledi�ini [1] yaz�s�nda g�rd�k. Programlama dilinde
yaz�lm��, i�inde \verb!if!, \verb!case!, hatta d�ng�ler bile i�erebilen
herhangi bir kod par�as�n�n t�revini alabilmemizi sa�layan otomatik t�rev
almak pek �ok alanda i�imize yarar. Optimizasyon alan� bunlar�n ba��nda
geliyor. D���n�rsek, e�er sembolik olarak t�rev almas� �ok �etrefil bir
durum varsa, tasaya gerek yok; bir fonksiyonu kodlayabildi�imiz anda onun
t�revini de alabiliriz demektir.



















\url{https://nikstoyanov.me/post/2019-04-14-numerical-optimizations/}

\url{http://kitchingroup.cheme.cmu.edu/blog/2018/10/10/Autograd-and-the-derivative-of-an-integral-function/}

\url{https://rlhick.people.wm.edu/posts/mle-autograd.html}

\url{http://kitchingroup.cheme.cmu.edu/blog/2018/11/03/Constrained-optimization-with-Lagrange-multipliers-and-autograd/}

\url{http://louistiao.me/notes/visualizing-and-animating-optimization-algorithms-with-matplotlib/}


[devam edecek]

Kaynaklar 

[1] Bayraml�, Ders Notlar�, {\em Otomatik T�rev Almak (Automatic Differentiation -AD-)}

\end{document}
